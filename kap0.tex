\section{Warum \LaTeX \  bzw. LUA\TeX ?}
\subsection{Vorteile von \LaTeX}
\begin{itemize}
  \item Tex Dateien kann man auch noch nach Jahren öffnen \\und in der Regel auch übersetzen.
  \item Keine Binärdateien. (Dokument auf Github direkt lesbar)
  \item Einfaches zusammenarbeiten (merge problemlos möglich)
  \item Nutzung mit Github und Travis CI relativ einfach
\end{itemize}
\subsection{Warum dann LUA\TeX ?}
\begin{itemize}
  \item Unterstützung von True-Type-Schriften \\(KEIN Zugriff auf Systemschriften !!!)
  \item komplexeres Programmieren innerhalb von Tex möglich.
\end{itemize}


\section{Warum GitHub und Travis-CI ?}

Im Prinzip könnte man jeden x beliebigen Server ... \\
Dort: \\
ein Git Repo + ssh und ein CI System. \\

Man könnte eines selbst zu Hause hosten und hätte die volle Kontrolle.\\

ABER:
Als Student möchte man vielleicht in Gruppen zusammen arbeiten um sich zusammen eine Mitschrift anzufertigen.
Nicht jeder möchte sich eine komplette \LaTeX - Installation an tun.\\

\"{}Och ne, iss mir zuviel Uffriss ...\"{} \\

Durfte ich mir schon oft anhören.






% \begin{itemize}
%   \item Tex Dateien kann man auch noch nach Jahren öffnen \\und in der Regel auch übersetzen.
%   \item Keine Binärdateien. (Dokument auf Github direkt lesbar)
%   \item Einfaches zusammenarbeiten (merge problemlos möglich)
%   \item Nutzung mit Github und Travis CI relativ einfach
% \end{itemize}
% \subsection{Warum dann LUA\TeX ?}
% \begin{itemize}
%   \item Unterstützung von True-Type-Schriften \\(KEIN Zugriff auf Systemschriften !!!)
%   \item komplexeres Programmieren innerhalb von Tex möglich.
% \end{itemize}
